% !TeX program = lualatex

\documentclass[a5paper, twoside]{memoir}

\usepackage{hyperref}
\usepackage{fontspec}
\usepackage[sfdefault]{plex-sans}
\usepackage{unicode-math}
\usepackage{microtype}

\newcommand*{\plogo}{\fbox{$\mathcal{PL}$}}

\newcommand*{\titleTH}{\begingroup % T&H Typography
\raggedleft
\vspace*{\baselineskip}
{\Large The Author}\\[0.167\textheight]
{\bfseries Manual of the}\\[\baselineskip]
{{\Huge F62/F66 Keyboard}}\\[\baselineskip]
{\small With 123 illustrations}\par
\vfill
%{\Large The Publisher \plogo}\par
\vspace*{3\baselineskip}
\endgroup}

\begin{document}
\pagestyle{empty}
\titleTH
\clearpage

\emph{(Copy paste of the text on the website for demonstration purposes)}

Brand New Model F Keyboards Manual (work in progress)

I’d appreciate it if those reading through the manual can offer additional content and/or corrections if something is not clear.   You can email me additional paragraphs and sentences!  Please do not copy and paste content from anywhere else unless you receive permission and attribute the source.

It is aimed at helping owners of both the original keyboards as well as the reproductions.

It will help provide buckling spring owners (everyone from beginners to advanced users) with a central place to learn key re-seating, disassembly, spring replacement, changing the layout, etc.

\section{Introduction to Model F Keyboards and Statement of Brand New Model F Keyboards Project Philosophy: Full user control of product maintenance and repairs}

    Check out Chyrosran22’s excellent Model F overview video:  \url{https://www.youtube.com/watch?v=y9Jds326gks}

    The Model F keyboard is a robust design. Every part is 100\% user-replaceable / user-repairable, often needing just a couple tools: screwdrivers, pliers, and at most a soldering iron. Compared to other consumer electronics products, Model F repair is easy and even a complete beginner can get up to speed quickly on how to use the keyboard software and keep their keyboard going for decades to come.

    This is a community type project where the goal is to have a product that you can use and learn to maintain yourself for decades from now, long after production has ended, with help from the community if need be. The most basic recommended maintenance involves just taking off the keys with a wire key cap puller to clean them with mild soap and water every now and then.

    The current state of low-quality manufactured goods encourages a culture of just throwing something away or bringing it back to the store if anything is wrong with it.  For something complicated like a motherboard or graphics card that’s probably the best option, but the new Model F project philosophy is for the users to be able to fix small issues themselves due to the simplicity and full repairability of the Model F design (many buyers come from the world of the original IBM keyboards that are decades old and almost certainly require some maintenance work, so they expect it and are used to it). This keeps costs down so I am able to offer these keyboards at less than half of what IBM charged for them (adjusted for inflation). Also there’s a great community of Model F keyboard fans, most prominently on sites like Deskthority and geekhack. You will never be out of reach of someone who can offer you advice and help in the coming years.

    There are definitely markets for hardware with service contracts. Not sure about a market for a +\$100 more costly keyboard with full maintenance and technical support though. IBM’s 1980s price guides mentioned they would require charging banking customers a minimum of about \$100 per year (not adjusted for inflation) for each original Model F keyboard in maintenance costs as part of a service contract. However, if I had to hire staff to deal with ``free'' returns, more personalized technical support / phone support, and doing even the most minor repairs (re-seating keys, replacing springs and barrels, changing the USB cable, etc.), each keyboard would cost a lot more because of overhead costs, and these keyboards are already not inexpensive to begin with. And it would slow me down even further mailing out these great keyboards. I believe that this direct to consumer, community type project is the best way to bring the Model F to as many people as possible and at the lowest possible cost.

    The best way you can help this project:  If you like using Model F and Model M buckling spring keyboards, the best way to help the project is to tell other ``tech-minded'' people you know about the project. That would be greatly appreciated, and only if you don’t mind. And please do share some photos of your newly set up Model F Keyboard on the Deskthority and geekhack project threads if you’d like!  The xwhatsit and QMK firmware options are both very powerful and you can do a lot of interesting things with them to customize your keyboard (check out the xwhatsit manual on \url{www.ModelFKeyboards.com/code}).

\end{document}

